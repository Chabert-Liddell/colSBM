% Options for packages loaded elsewhere
\PassOptionsToPackage{unicode}{hyperref}
\PassOptionsToPackage{hyphens}{url}
%
\documentclass[
]{article}
\usepackage{lmodern}
\usepackage{amssymb,amsmath}
\usepackage{ifxetex,ifluatex}
\ifnum 0\ifxetex 1\fi\ifluatex 1\fi=0 % if pdftex
  \usepackage[T1]{fontenc}
  \usepackage[utf8]{inputenc}
  \usepackage{textcomp} % provide euro and other symbols
\else % if luatex or xetex
  \usepackage{unicode-math}
  \defaultfontfeatures{Scale=MatchLowercase}
  \defaultfontfeatures[\rmfamily]{Ligatures=TeX,Scale=1}
\fi
% Use upquote if available, for straight quotes in verbatim environments
\IfFileExists{upquote.sty}{\usepackage{upquote}}{}
\IfFileExists{microtype.sty}{% use microtype if available
  \usepackage[]{microtype}
  \UseMicrotypeSet[protrusion]{basicmath} % disable protrusion for tt fonts
}{}
\makeatletter
\@ifundefined{KOMAClassName}{% if non-KOMA class
  \IfFileExists{parskip.sty}{%
    \usepackage{parskip}
  }{% else
    \setlength{\parindent}{0pt}
    \setlength{\parskip}{6pt plus 2pt minus 1pt}}
}{% if KOMA class
  \KOMAoptions{parskip=half}}
\makeatother
\usepackage{xcolor}
\IfFileExists{xurl.sty}{\usepackage{xurl}}{} % add URL line breaks if available
\IfFileExists{bookmark.sty}{\usepackage{bookmark}}{\usepackage{hyperref}}
\hypersetup{
  pdftitle={Analyzing the capacity of the colBiSBM to recover structure for missing data from other networks},
  hidelinks,
  pdfcreator={LaTeX via pandoc}}
\urlstyle{same} % disable monospaced font for URLs
\usepackage[margin=1in]{geometry}
\usepackage{graphicx}
\makeatletter
\def\maxwidth{\ifdim\Gin@nat@width>\linewidth\linewidth\else\Gin@nat@width\fi}
\def\maxheight{\ifdim\Gin@nat@height>\textheight\textheight\else\Gin@nat@height\fi}
\makeatother
% Scale images if necessary, so that they will not overflow the page
% margins by default, and it is still possible to overwrite the defaults
% using explicit options in \includegraphics[width, height, ...]{}
\setkeys{Gin}{width=\maxwidth,height=\maxheight,keepaspectratio}
% Set default figure placement to htbp
\makeatletter
\def\fps@figure{htbp}
\makeatother
\setlength{\emergencystretch}{3em} % prevent overfull lines
\providecommand{\tightlist}{%
  \setlength{\itemsep}{0pt}\setlength{\parskip}{0pt}}
\setcounter{secnumdepth}{-\maxdimen} % remove section numbering

\title{Analyzing the capacity of the colBiSBM to recover structure for
missing data from other networks}
\author{}
\date{\vspace{-2.5em}}

\begin{document}
\maketitle

\begin{verbatim}
## Le chargement a nécessité le package : ggplot2
\end{verbatim}

\begin{verbatim}
## Le chargement a nécessité le package : tidyverse
\end{verbatim}

\begin{verbatim}
## -- Attaching core tidyverse packages ------------------------ tidyverse 2.0.0 --
## v dplyr     1.1.2     v readr     2.1.4
## v forcats   1.0.0     v stringr   1.5.0
## v lubridate 1.9.2     v tibble    3.2.1
## v purrr     1.0.1     v tidyr     1.3.0
## -- Conflicts ------------------------------------------ tidyverse_conflicts() --
## x dplyr::filter() masks stats::filter()
## x dplyr::lag()    masks stats::lag()
## i Use the conflicted package (<http://conflicted.r-lib.org/>) to force all conflicts to become errors
\end{verbatim}

\begin{verbatim}
## `summarise()` has grouped output by 'prop_NAs'. You can override using the
## `.groups` argument.
\end{verbatim}

\hypertarget{contexte-des-simulations}{%
\section{Contexte des simulations}\label{contexte-des-simulations}}

The idea is to benchmark the capacity of the models when NAs are in the
data.

To do this, whe choose the below structure: ! PARAMETERS OF THE
SIMULATION !

And set some randomly chosen interactions to NA. The below plots will
show the different quality indicators in function of proportion of NAs
in the first of the 3 networks.

\hypertarget{auc-in-function-of-the-proportion-of-nas}{%
\section{AUC in function of the proportion of
NAs}\label{auc-in-function-of-the-proportion-of-nas}}

\includegraphics{NA_robustness_analyse_files/figure-latex/auc_plots-1.pdf}

\begin{figure}
\centering
\includegraphics{NA_robustness_analyse_files/figure-latex/ARI_row_plot-1.pdf}
\caption{Difference of ARI for the row clusterings}
\end{figure}

\begin{figure}
\centering
\includegraphics{NA_robustness_analyse_files/figure-latex/ARI_col_plot-1.pdf}
\caption{Difference of ARI for the columns clusterings}
\end{figure}

\end{document}
